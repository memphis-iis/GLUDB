\documentclass[t,handout]{beamer}

\usepackage[symbol*]{footmisc}
\DefineFNsymbolsTM{myfnsymbols}{% def. from footmisc.sty "bringhurst" symbols
  \textasteriskcentered *
}%
\setfnsymbol{myfnsymbols}

\usepackage{verbatim}
\usepackage{beamerthemeCopenhagen}
\beamertemplatenavigationsymbolsempty
\usepackage{amsmath}

\title{Dist, Docs, and Testing with OSS on GitHub}
\author{Craig Kelly}
\date{\today}

\begin{document}

\frame{\titlepage}

\section{Introduction}
    \subsection{Me and GLUDB}
        \frame {
            \frametitle{Hello}
            I'm Craig, a Research Developer for the Institute for Intelligent Systems
            at the University of Memphis (http://www.memphis.edu/iis/).

            \vspace{\baselineskip}

            Our mission depends on an interdisciplinary approach that brings together
            researchers from many different research areas in the cognitive sciences,
            including biology, communication sciences and disorders, computer science,
            education, engineering, linguistics, philosophy, physics, and psychology.
        }

        \frame {
            \frametitle{Why Does This Matter?}
            Grad students!

            \vspace{\baselineskip}

            They generally aren't experienced coders. In some cases they are just
            learning to program. And they're generally more concerned with DOE
            than making the database work. So...
        }

        \frame {
            \frametitle{What is GLUDB?}
            (And why should you care?)
            \begin{itemize}
                \item Simplified for the demands of working academics and grad students
                \item Annotate and go!
                \item Active Record (-ish)
                \item Documented-oriented data store with support for indexing
                \item Includes optional functionality for object change history and backups
                \item Allows easily moving from dev to server to cloud
                \item Supports sqlite, MongoDB, Google Cloud Datastore, and Amazon DynamoDB
            \end{itemize}
        }

        \frame {
            \frametitle{Is GLUDB for you?}
            Probably not.

            \vspace{\baselineskip}

            If you want a good, solid ORM then use SQLAlchemy. Or Django's ORM
            is you're Django'ing. Especially if you care about migrations.

            \vspace{\baselineskip}

            GLUDB can be useful for rapid iteration, hobby projects, or when
            you need a nice ``object store''
        }

    \subsection{Overview}
        \frame {
            \frametitle{Dist, Docs, and Testing}
            \begin{itemize}
                \item Dist via PiPy (pipy.python.org)
                \item Docs via Read The Docs (readthedocs.org)
                \item Testing via Travis CI
            \end{itemize}
        }

\section{Read The Docs}
    \subsection{Overview}
        \frame {
            \frametitle{Overview}
            \begin{itemize}
                \item Free account!
                \item Pulls from your repo automatically when you push changes
                \item Supports different doc types
                \item Has a nice template, and provides downloads
            \end{itemize}
        }
    \subsection{Implementation}
        \frame {
            \frametitle{How to Implement}
            \begin{itemize}
                \item You write docs in reStructuredText or Markdown
                \item You push the docs to your GitHub repo
                \item RTD pull the docs, builds them, and shows them on the web
            \end{itemize}
        }
    \subsection{Example from GLUDB}
        \frame {
            \frametitle{Working Example}
            \begin{itemize}
                \item We used Markdown - you should probably use Sphinx
                \item See our mkdocs.yml and our docs directory
                \item Published at http://gludb.readthedocs.org/en/latest/
            \end{itemize}
        }

\section{Travis CI}
    \subsection{Overview}
        \frame {
            \frametitle{Overview}
            \begin{itemize}
                \item Continuous Integration for free! (for open source repos on GitHub)
                \item Pulls from your repo automatically when you push changes
                \item Multiple environments (e.g. test both Python 2 and 3)
                \item Supports lots of languages
                \item Provides lots of services for testing (http://docs.travis-ci.com/user/database-setup/)
                \item Allows for container customization if what need isn't in the box
            \end{itemize}
        }
    \subsection{Implementation}
        \frame {
            \frametitle{How to Implement}
            \begin{itemize}
                \item You have tests to run!
                \item You log in to travis-ci.org with your GitHub account
                \item You set up your repository
                \item You create a .travis.yml file
                \item When you push to your repository your tests are run
            \end{itemize}
        }
    \subsection{Example from GLUDB}
        \frame {
            \frametitle{Working Example}
            \begin{itemize}
                \item See our travis.yml and tests
                \item We run tests for both Python 2.7 and 3.4
                \item We use the Travis CI MongoDB service
                \item We use a manual process to run backing services that we
                      need using supervisor. See start\_ci\_services.sh
                      (See next slide)
            \end{itemize}
        }

        \frame {
            \frametitle{Test Services Startup Script}
            \begin{itemize}
                \item Uses npm to install the dynalite server
                \item Uses wget to download and extract the gcd test server
                \item Uses virtualenv and pip to set up a Python 2.7 environment for supervisor and our S3 test server
                \item We even run a Python script as a service (s3server.py) that we copied from Facebook and modified!
                \item This is all very cool, but also check out our sleep hack :)
            \end{itemize}
        }

\section{PyPI}
    \subsection{Overview}
        \frame {
            \frametitle{Overview}
            Do you want to do \texttt{pip install my-cool-package}?

            \vspace{\baselineskip}

            Then you need to be on PyPI (https://pypi.python.org/pypi)

            \vspace{\baselineskip}

            Not to be confused with the PyPy project at pypy.org
        }
    \subsection{Implementation}
        \frame {
            \frametitle{How to Implement}
            \begin{itemize}
                \item You get your account at pypi.python.org
                \item You create a setup script for your package
                \item You upload to PyPI
            \end{itemize}
        }
    \subsection{Example from GLUDB}
        \frame {
            \frametitle{Working Example}
            \begin{itemize}
                \item First we need a steup script - setup.py (and setup.cfg)
                      that includes dependencies. (IMPORTANT: your requirements.txt
                      file will not get used)
                \item It also helps to have a wrapper script like build.sh
                \item You can handle uploads using twine
                \item If you like Markdown for your README (who doesn't), you
                      can use pandoc for conversion
                \item Note that you probably won't need the extras
            \end{itemize}
        }

\section{Conclusion}
    \subsection{Conclusion}
        \frame {
            \frametitle{Don't Panic}
            RTD and Travis have excellent documentation (and PyPI's isn't bad).

            \vspace{\baselineskip}

            They're also well-covered at StackOverflow and the usual places
        }
        \frame {
            \frametitle{Thank You!}
            Thanks for listening
        }

\end{document}
